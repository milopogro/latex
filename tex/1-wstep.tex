\newpage % Rozdziały zaczynamy od nowej strony.
\section{Wstęp}
W dzisiejszym świecie, w którym technologia informatyczna przenika niemal każdy aspekt działalności biznesowej, umiejętne zarządzanie projektami IT staje się kluczowe dla osiągnięcia sukcesu. Złożoność tych projektów, obejmujących różnorodne technologie, narzędzia i zespoły specjalistów, stawia przed kierownikami projektów wysokie wymagania kompetencyjne. W tym kontekście zasadne jest pytanie, czy wiedza specjalistyczna z dziedziny inżynierii oprogramowania jest niezbędnym elementem w zestawie umiejętności efektywnego kierownika projektu informatycznego. 

Literatura z zakresu zarządzania projektami IT wskazuje na wielowymiarowość kompetencji wymaganych od kierowników projektów. Kluczowe znaczenie mają zarówno umiejętności techniczne, jak i kompetencje miękkie. Do umiejętności technicznych zaliczamy między innymi znajomość szerokiego spektrum technologii IT, doświadczenie w pracy z systemami informatycznymi oraz umiejętność zarządzania zespołem technicznym. \autocite{haggerty}\autocite{langer} Natomiast kompetencje miękkie, takie jak komunikacja, negocjacje, budowanie relacji i zarządzanie konfliktami, są niezbędne do efektywnej współpracy z zespołem projektowym, klientami i dostawcami.

Wiedza specjalistyczna z dziedziny inżynierii oprogramowania może być niewątpliwie atutem w pracy kierownika projektu informatycznego. Umożliwia ona lepsze zrozumienie technicznych aspektów projektu, trafniejszy wybór odpowiednich narzędzi i technologii, a także efektywniejszą komunikację z zespołem programistów. Hobbs i inni zauważają jednak, że po osiągnięciu pewnego podstawowego poziomu wiedzy, jej dalsze pogłębianie nie przekłada się automatycznie na wzrost kompetencji kierownika projektu. \autocite{hobbs} Kluczem do sukcesu w zarządzaniu projektami IT może być zatem znalezienie optymalnej równowagi między umiejętnościami technicznymi a miękkimi, dostosowanej do specyfiki danego projektu i kontekstu organizacyjnego.

Należy również wziąć pod uwagę zróżnicowanie typów projektów IT. W zależności od rodzaju projektu, na przykład rozwoju oprogramowania, wdrożenia systemu ERP czy utrzymania infrastruktury IT, znaczenie poszczególnych kompetencji może się różnić. Badania wskazują, że w przypadku projektów o dużej złożoności kluczowe znaczenie mają kompetencje miękkie, takie jak umiejętność zarządzania zespołem, rozwiązywania konfliktów i budowania relacji. W projektach o mniejszej złożoności większą rolę mogą odgrywać umiejętności techniczne, w tym wiedza programistyczna. \autocite{podgorska}\autocite{jiang}

Pytanie o niezbędność wiedzy technicznej dla kierownika projektu IT jest niezwykle istotne, ponieważ ma bezpośredni wpływ na dobór kandydatów na to stanowisko, a co za tym idzie, na efektywność i sukces realizowanych projektów. Do weryfikacji prac poruszających tę tematykę zastosowano metodykę przeglądu zakresu literatury, czyli przegląd literatury, który ma na celu zidentyfikowanie zakresu dostępnych badań na dany temat, a nie syntezę wyników badań. \autocite{metodyka} Przegląd ten wykazał, że w dostępnej literaturze brakuje prac, które odpowiedziałyby w sposób bezpośredni na pytanie stawiane w tytule tej pracy. Dostępne źródła skupiają się najczęściej na ogólnych umiejętnościach kierownika projektu, nie zgłębiając zbytnio tematu inżynierii oprogramowania lub jedynie na pewnej gałęzi projektów informatycznych. \autocite{data} Brak jasnej odpowiedzi na to pytanie generuje wiele dyskusji i kontrowersji w środowisku zarządzania projektami.

Celem niniejszej pracy magisterskiej jest dogłębna analiza argumentów za i przeciw tezie, że kierownik projektu informatycznego powinien posiadać wiedzę z zakresu inżynierii oprogramowania. W pracy zostaną przeanalizowane role i zadania kierownika projektu w kontekście specyfiki projektów IT, ze szczególnym uwzględnieniem unikalnych wyzwań i technologii stosowanych w tej dziedzinie. Zostaną omówione popularne metodyki zarządzania projektami informatycznymi, takie jak tradycyjne metody kaskadowe oraz zwinne metodyki Agile i Scrum, aby ocenić, w jaki sposób wpływają one na rolę i kompetencje kierownika projektu. Przeprowadzone zostanie badanie ankietowe wśród pracowników z branży IT, aby poznać ich opinie na temat niezbędnego wykształcenia i doświadczenia dla kierowników projektów informatycznych. Na podstawie analizy literatury przedmiotu, studiów przypadków oraz wyników badania ankietowego, praca sformułuje wnioski dotyczące znaczenia wiedzy technicznej dla kierownika projektu informatycznego.