\newpage

\section*{Rozdział X: Analiza badania dotyczącego kompetencji kierownika projektu informatycznego}

\subsection*{1. Cel badania}

Celem przeprowadzonego badania było zidentyfikowanie kluczowych kompetencji, które według specjalistów z branży IT powinien posiadać skuteczny kierownik projektu informatycznego. Inspirację do konstrukcji narzędzia badawczego stanowiła klasyfikacja kompetencji zaproponowana w pracy \textcite{araujo2020it}, gdzie autorki wyróżniły pięć głównych obszarów: interpersonalne, przywódcze, strategiczne, organizacyjne oraz techniczne.

\subsection*{2. Metodologia}

Badanie zostało zrealizowane w formie anonimowej ankiety internetowej. Wzięło w nim udział 30 respondentów reprezentujących różne stanowiska w sektorze IT. Kwestionariusz zawierał pytania zamknięte z oceną w skali Likerta (1–5), a także pytania otwarte, umożliwiające rozszerzenie odpowiedzi o opinie jakościowe.

\subsection*{3. Charakterystyka próby}

Wśród respondentów dominowali programiści (20 osób), następnie kierownicy projektów (5 osób) oraz inne role (5 osób). Największa część respondentów miała od 2 do 10 lat doświadczenia zawodowego. Większość uczestników brała udział w projektach rozwojowych (25 osób) oraz utrzymaniowych (24 osoby), a 15 wskazało na doświadczenie we wdrożeniach.

\subsection*{4. Wyniki ilościowe}

\subsubsection*{4.1 Porównanie średnich ocen kompetencji}

\begin{table}[H]
\centering
\caption{Średnie oceny ważności kategorii kompetencji}
\begin{tabular}{ll}
\toprule
\textbf{Kategoria kompetencji} & \textbf{Średnia ocena (1–5)} \\
\midrule
Interpersonalne i komunikacyjne          & \textbf{4{,}71} \\
Przywódcze i zarządcze                   & 4{,}61 \\
Strategiczne i analityczne               & 4{,}53 \\
Finansowe i organizacyjne                & 4{,}41 \\
Personalne                               & 4{,}39 \\
Techniczne (inżynieria oprogramowania)   & 4{,}18 \\
\bottomrule
\end{tabular}
\end{table}

Na podstawie powyższych danych najwyżej oceniono kompetencje interpersonalne i komunikacyjne, co potwierdza ich fundamentalne znaczenie w roli kierownika projektu IT.

\begin{figure}[H]
\centering
\includegraphics[width=0.8\textwidth]{kompetencje_slupki.png}
\caption{Średnie oceny kompetencji według kategorii}
\end{figure}

\subsubsection*{4.2 Analiza szczegółowa}

W obszarze kompetencji interpersonalnych najwyżej oceniono umiejętność komunikowania się z zespołem, a także empatię i zdolność rozwiązywania konfliktów. W kompetencjach przywódczych dominowały takie cechy jak zdolność motywowania, delegowania oraz zarządzanie czasem.

Kompetencje strategiczne obejmowały myślenie analityczne, podejmowanie decyzji i zarządzanie ryzykiem — wszystkie uzyskały wysokie noty. Kompetencje techniczne, mimo niższych ocen ogólnych, zostały uznane za istotne w kontekście określonych ról projektowych.

\begin{figure}[H]
\centering
\includegraphics[width=0.8\textwidth]{kompetencje_radar.png}
\caption{Porównanie kompetencji według średnich wartości (wykres radarowy)}
\end{figure}

\subsection*{5. Wyniki jakościowe}

Respondenci wskazali, że brak wiedzy technicznej u kierownika może prowadzić do:

\begin{itemize}
  \item opóźnień harmonogramowych,
  \item trudności w koordynacji zadań,
  \item nieporozumień między zespołem a kierownikiem.
\end{itemize}

Jednocześnie, nadmierne skupienie na aspektach technicznych może skutkować:

\begin{itemize}
  \item mikrozarządzaniem,
  \item brakiem perspektywy biznesowej,
  \item paraliżem decyzyjnym.
\end{itemize}

\subsection*{6. Wnioski}

Badanie potwierdziło hipotezy sformułowane na podstawie literatury — kompetencje społeczne (interpersonalne i przywódcze) mają kluczowe znaczenie dla skutecznego zarządzania projektami IT. Jednocześnie nie można ignorować roli wiedzy technicznej, która w odpowiednich proporcjach wspiera procesy zarządcze, komunikację z zespołem oraz rozumienie ryzyka projektowego.

\printbibliography

\end{document}

